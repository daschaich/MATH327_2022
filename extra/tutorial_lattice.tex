% ------------------------------------------------------------------
\documentclass[12 pt]{article} % A4 paper set by geometry package below
\pagenumbering{arabic}
\setlength{\parindent}{10 mm}
\setlength{\parskip}{12 pt}

% Nimbus Sans font should be reasonably legible
\usepackage{helvet}
\renewcommand{\familydefault}{\sfdefault}
\usepackage[T1]{fontenc}  % Without this \textsterling produces $

% Section header spacing
\usepackage{titlesec}
\titlespacing\section{0pt}{12pt plus 4pt minus 2pt}{0pt plus 2pt minus 2pt}
\titlespacing\subsection{0pt}{12pt plus 4pt minus 2pt}{0pt plus 2pt minus 2pt}
\titlespacing\subsubsection{0pt}{12pt plus 4pt minus 2pt}{0pt plus 2pt minus 2pt}

\usepackage{amsmath}
\usepackage{amssymb}
\usepackage{graphicx}
\usepackage{verbatim}    % For comment
\usepackage[paper=a4paper, marginparwidth=0 cm, marginparsep=0 cm, top=2.5 cm, bottom=2.5 cm, left=3 cm, right=3 cm, includemp]{geometry}
\usepackage[pdftex, pdfstartview={FitH}, pdfnewwindow=true, colorlinks=true, citecolor=blue, filecolor=blue, linkcolor=blue, urlcolor=blue, pdfpagemode=UseNone]{hyperref}

% Put module code and last-modified date in footer
\usepackage{fancyhdr}
\pagestyle{fancy}
\fancyhf{}
\renewcommand{\headrulewidth}{0pt}
\cfoot{{\small \thisunit}\hfill \thepage\hfill {\small \moddate}}

% Hopefully address Canvas complaints about pdf tagging
%\usepackage[tagged]{accessibility}
\hypersetup {
  pdfauthor={David Schaich},
  pdftitle={Statistical Physics Tutorial Problem},
}
% ------------------------------------------------------------------



% ------------------------------------------------------------------
% Shortcuts
\newcommand{\be}{\ensuremath{\beta} }
\newcommand{\eps}{\ensuremath{\varepsilon} }
\newcommand{\om}{\ensuremath{\omega} }
\newcommand{\pderiv}[2]{\ensuremath{\frac{\partial #1}{\partial #2}} }
% ------------------------------------------------------------------



% ------------------------------------------------------------------
\begin{document}
\newcommand{\thisunit}{MATH327 Tutorial (Lattice)}
\newcommand{\moddate}{Last modified 29 Apr.~2022}
\begin{center}
  {\Large \textbf{MATH327: Statistical Physics, Spring 2022}} \\[12 pt]
  {\Large \textbf{Tutorial problem \ --- \ Lattices}} \\[24 pt]
\end{center}

In the module we are focusing on simple cubic lattices with periodic boundary conditions, but other lattice structures play important roles in both nature and mathematics.
Some of the remarkable electronic properties of graphene, for example, are due to its two-dimensional honeycomb lattice structure, while more elaborate three-dimensional lattices play central roles in the search for materials exhibiting high-temperature superconductivity.

The figure below shows three simple two-dimensional lattices, each of which has a different \textbf{coordination number} --- the number of nearest neighbours for each site (with periodic boundary conditions). % TODO: Adapted from doi.org/10.1007/978-3-319-10924-4_6
We have already seen that the square lattice has coordination number $C = 2d = 4$, and generalizes to simple cubic and hyper-cubic lattices in higher dimensions.

\begin{center}\includegraphics[width=\textwidth]{figs/lattices.pdf}\end{center}

The honeycomb lattice of graphene has a smaller coordination number $C = d + 1 = 3$, and generalizes to `hyper-diamond' lattices in higher dimensions.
Finally, the triangular lattice essentially fills in the middle of each honeycomb cell, leading to coordination number $C = 2(d + 1) = 6$.
Its higher-dimensional generalizations are known as $A_d^*$ lattices, of which the simplest example is the three-dimensional body-centered cubic lattice shown below.
Also shown below is the `kagome' lattice, which has the same $C = 4$ as the square lattice, illustrating that the coordination number is insufficient to completely characterize a lattice.

\begin{center}\includegraphics[width=0.4\textwidth]{figs/bcc.pdf}\hspace{0.15\textwidth}\includegraphics[width=0.4\textwidth]{figs/kagome.pdf}\end{center}

Things become more interesting when we consider generalizing the Ising model to have energy
\begin{equation*}
  E = -J \sum_{(jk)} s_j s_k - H \sum_{n = 1}^N s_n,
\end{equation*}
with nearest neighbours $(jk)$ defined by any of the three simple two-dimensional lattices above.
While any positive interaction strength $J > 0$ can be rescaled to $J = 1$ without loss of generality, the case of a constant negative $J < 0$ is qualitatively different.
Specializing to $d = 2$ and $H = 0$, let's consider a couple of \textbf{conceptual questions}:
What are the minimum-energy ground states for each case $J > 0$ and $J < 0$, for each of the square, honeycomb and triangular lattices?
Can you think of an order parameter distinguishing these ground states from the disordered micro-states that dominate at high temperatures?

Generalizing the Ising model in this way opens up a vast landscape of possible applications both practical and abstract.
As one example (with both practical and abstract relevance), a \textbf{spin glass} can be modeled by allowing the interaction strength to vary from site to site,
\begin{equation*}
  E_{\text{SG}} = -\sum_{(jk)} J_{jk} s_j s_k.
\end{equation*}
Giorgio Parisi was awarded part of the 2021 Nobel Prize in Physics for his work studying the mathematics of such spin glass systems.
In particular, he was able to solve the so-called Sherrington--Kirkpatrick model
\begin{equation*}
  E_{\text{SG}} = -\sum_{j < k} J_{jk} s_j s_k,
\end{equation*}
where the values $J_{jk}$ are randomly drawn from a gaussian distribution around some mean $J_0$ and the system is defined on a fully connected lattice (or \textbf{complete graph}) where every site $j$ is a nearest neighbour of every other site $k \ne j$.
(Summing only over $j < k$ and not $j > k$ avoids double-counting the link $jk$.)
As shown on the next page, a fully connected lattice with $N$ sites has $\frac{1}{2} N (N - 1) = \binom{N}{2}$ links.

While we may look a bit more closely at spin glasses if time permits during the next couple of weeks, for now let's consider a simpler Ising system with constant interaction strength on the fully connected lattice:
\begin{equation*}
  E = -\frac{J}{N} \sum_{j < k} s_j s_k - H \sum_{n = 1}^N s_n.
\end{equation*}
We normalize the interaction strength by $N$ so that the system retains a finite energy per spin in the $N \to \infty$ thermodynamic limit.

Can you compute a closed-form expression for the partition function of this Ising model on the fully connected lattice?
As a hint, it may be profitable to reorganize the calculation into a sum over the $N + 1$ possible values of the magnetization $-1 \leq m \leq 1$, and counting how many micro-states there are with a given magnetization.
(This is analogous to the fugacity expansion that reorganizes the grand-canonical partition function into a sum over possible particle numbers.)
The energy above would need to be rewritten in terms of the magnetization, which is easier when summing over all $j < k$ compared to considering only nearest neighbours.
Finally, for large $N$ we can approximate the $N + 1$ possible values of $m$ as continuously varying, and integrate
\begin{equation*}
  Z = \int_{-1}^1 \ (\cdots) \ dm.
\end{equation*}

\begin{center}\includegraphics[width=\textwidth]{figs/completeGraph.pdf}\end{center}

\end{document}
% ------------------------------------------------------------------
