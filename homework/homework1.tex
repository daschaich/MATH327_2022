% ------------------------------------------------------------------
\documentclass[12 pt]{article} % A4 paper set by geometry package below
\pagenumbering{arabic}
\setlength{\parindent}{10 mm}
\setlength{\parskip}{12 pt}

% Nimbus Sans font should be reasonably legible
\usepackage{helvet}
\renewcommand{\familydefault}{\sfdefault}
\usepackage[T1]{fontenc}  % Without this \textsterling produces $

% Section header spacing
\usepackage{titlesec}
\titlespacing\section{0pt}{12pt plus 4pt minus 2pt}{0pt plus 2pt minus 2pt}
\titlespacing\subsection{0pt}{12pt plus 4pt minus 2pt}{0pt plus 2pt minus 2pt}
\titlespacing\subsubsection{0pt}{12pt plus 4pt minus 2pt}{0pt plus 2pt minus 2pt}

\usepackage{amsmath}
\usepackage{amssymb}
\usepackage{graphicx}
\usepackage{verbatim}    % For comment
\usepackage[paper=a4paper, marginparwidth=0 cm, marginparsep=0 cm, top=2.5 cm, bottom=2.5 cm, left=3 cm, right=3 cm, includemp]{geometry}
\usepackage[pdftex, pdfstartview={FitH}, pdfnewwindow=true, colorlinks=true, citecolor=blue, filecolor=blue, linkcolor=blue, urlcolor=blue, pdfpagemode=UseNone]{hyperref}

% Put module code and last-modified date in footer
\usepackage{fancyhdr}
\pagestyle{fancy}
\fancyhf{}
\renewcommand{\headrulewidth}{0pt}
\cfoot{{\small \thisweek}\hfill \thepage\hfill {\small \moddate}}

% Hopefully address Canvas complaints about pdf tagging and title
%\usepackage[tagged]{accessibility}
\hypersetup {
  pdfauthor={David Schaich},
  pdftitle={Statistical Physics Homework},
}
% ------------------------------------------------------------------



% ------------------------------------------------------------------
% Shortcuts
\newcommand{\be}{\ensuremath{\beta} }
\newcommand{\De}{\ensuremath{\Delta} }
\newcommand{\eps}{\ensuremath{\varepsilon} }
\newcommand{\si}{\ensuremath{\sigma} }
\newcommand{\vdr}{\ensuremath{v_{\mathrm{dr}}} }
\newcommand{\vev}[1]{\ensuremath{\left\langle #1 \right\rangle} }
\newcommand{\pderiv}[2]{\ensuremath{\frac{\partial #1}{\partial #2}} }
\newcommand{\showmarks}[1]{\rightline{\texttt{[#1 marks]}}} % \showmarks needs to follow a blank line!
\newcommand{\TODO}[1]{\textcolor{red}{\textbf{#1}}}
% ------------------------------------------------------------------



% ------------------------------------------------------------------
\begin{document}
\newcommand{\thisweek}{MATH327 Homework 1}
\newcommand{\moddate}{Last modified 21 Feb.~2022}
\begin{center}
  {\Large \textbf{MATH327: Statistical Physics, Spring 2022}} \\[12 pt]
  {\Large \textbf{Homework assignment 1}} \\[24 pt]
\end{center}

\section*{Instructions}
Complete all four questions below and submit your solutions by file upload \href{https://liverpool.instructure.com/courses/47333/assignments/178542}{on Canvas}.\footnote{By submitting solutions to this assessment you affirm that you have read and understood the \href{https://www.liverpool.ac.uk/media/livacuk/tqsd/code-of-practice-on-assessment/appendix_L_cop_assess.pdf}{Academic Integrity Policy} detailed in Appendix L of the Code of Practice on Assessment and have successfully passed the Academic Integrity Tutorial and Quiz.  The marks achieved on this assessment remain provisional until they are ratified by the Board of Examiners in June 2022.}
Clear and neat presentations of your workings and the logic behind them will contribute to your mark.
This assignment is \textbf{due by 23:59 on Thursday, 3 March}, and anonymous marking is turned on.
% ------------------------------------------------------------------



% ------------------------------------------------------------------
\section*{Question 1: Drift and diffusion}
There has been an oil spill $1~\mathrm{km}$ away from a \href{https://en.wikipedia.org/wiki/Marine_protected_area}{marine protected area} (MPA).
$1000$ tonnes of oil are in the water.
You have been called on to estimate how much time is available for the government to take action to protect the MPA.
The motion of each oil droplet can be modeled as a one-dimensional random walk towards (or away from) the MPA, with diffusion constant $D = 2~\mathrm{m}/\sqrt{\mathrm{min}}$.
If the oil has a drift velocity $\vdr = 1~\mathrm{m}/\mathrm{min}$ towards the MPA, how long do we have before $10$~tonnes of oil is inside the MPA?
How much additional time will it take for the amount of oil inside the MPA to double to a total of $20$~tonnes?

\showmarks{8}

If the oil were washing up on a shore instead of entering an MPA, the mathematics would be significantly more complicated, because each droplet's random walk would \textit{stop} once it reached the shore and left the water.
This is known as a \textit{first-passage process}.
Without attempting this more complicated calculation, explain whether it will take more time, less time or the same amount time for $10$ tonnes of the spilled oil to wash up on shore, compared to the MPA case considered above, with everything else the same.

\showmarks{2}

\textbf{Hint:} \href{https://scipy.org}{SciPy} is one tool you can use to invert the error function
\begin{equation*}
  \mathrm{erf}(u) = \frac{1}{\sqrt{\pi}} \int_{-u}^u e^{-x^2} \; dx \equiv P
\end{equation*}
and find the $u \geq 0$ corresponding to a given $0 \leq P < 1$.
There is example code and output on the next page\dots
\newpage
\begin{verbatim}
>>> import math
>>> from scipy import special
>>>
>>> sigmas = [0.682689492, 0.954499736, 0.997300204, \
...                        0.999936656, 0.999999427]
>>> for P in sigmas:
...   u = special.erfinv(P)
...   n = round(u * math.sqrt(2.0))
...   print("u = %.4f for P=%.7f (%d sigma)" % (u, P, n))

u = 0.7071 for P=0.6826895 (1 sigma)
u = 1.4142 for P=0.9544997 (2 sigma)
u = 2.1213 for P=0.9973002 (3 sigma)
u = 2.8284 for P=0.9999367 (4 sigma)
u = 3.5356 for P=0.9999994 (5 sigma)
\end{verbatim}
% ------------------------------------------------------------------



% ------------------------------------------------------------------
\vfill
\section*{Question 2: Negative temperature}
Consider a system of $N$ distinguishable particles in which the energy of each particle can assume only two distinct values, $0$ and $\eps$.
Denote by $n_0$ the number of particles that have energy $0$, and by $n_1 = N - n_0$ the number of particles have energy $\eps$.
Assume both $n_0 \gg 1$ and $n_1 \gg 1$.

Suppose the system is isolated and governed by the micro-canonical ensemble with conserved total energy $E$.
Approximating $\log(n!) \approx n\log n - n$, what is the entropy of the system in terms of $N$, $E$ and $\eps$?

\showmarks{4}

What is the temperature $T$ of the system in terms of $N$, $E$ and $\eps$?
Show that $T$ can be negative.

\showmarks{4}

What happens when a system of negative temperature is brought into thermal contact with a system of positive temperature?

\showmarks{4}
% ------------------------------------------------------------------



% ------------------------------------------------------------------
\newpage
\section*{Question 3: Heat capacity}
Starting from the average internal energy for the canonical ensemble,
\begin{equation*}
  \vev{E} = \frac{1}{Z} \sum_{i = 1}^M E_i \, e^{-\be E_i},
\end{equation*}
derive a relation between the heat capacity
\begin{equation*}
  c_v = \pderiv{}{T} \vev{E}
\end{equation*}
and the quantity $\vev{\left(E - \vev{E}\right)^2}$.

\showmarks{4}

If the heat capacity vanishes at finite temperature, what can we conclude about the micro-state energies $E_i$?

\showmarks{2}

Derive a relation between $c_v$, $\pderiv{}{T} c_v$ and the quantity $\vev{\left(E - \vev{E}\right)^3}$.

\showmarks{6}
% ------------------------------------------------------------------



% ------------------------------------------------------------------
\vfill
\section*{Question 4: Indistinguishable spins}
The Helmholtz free energy for $N \gg 1$ indistinguishable spins is
\begin{equation*}
  F_I(\be) = -NH - \frac{\log\left[1 - e^{-2(N + 1) \be H}\right]}{\be} + \frac{\log\left[1 - e^{-2 \be H}\right]}{\be}.
\end{equation*}
What are the corresponding internal energy $\vev{E}_I$ and entropy $S_I$?

\showmarks{4}

What are the two leading terms in each low-temperature ($e^{-2\be H} \ll 1$) expansion of $\vev{E}_I$ and $S_I$?

\showmarks{6}

What is the leading term in the high-temperature ($\be H \ll 1$) expansion of $\vev{E}_I$?
What are the two leading terms in the high-temperature expansion of $S_I$?

\showmarks{6}
% ------------------------------------------------------------------



% ------------------------------------------------------------------
\end{document}
% ------------------------------------------------------------------
