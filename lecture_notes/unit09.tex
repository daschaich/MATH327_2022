% ------------------------------------------------------------------
\renewcommand{\thisweek}{MATH327 Week 9}
\renewcommand{\moddate}{Last modified 24 Apr.~2021}
\setcounter{section}{9}
\setcounter{subsection}{0}
\phantomsection
\addcontentsline{toc}{section}{Week 9: Quantum gases of fermions}
\section*{Week 9: Quantum gases of fermions}
\subsection{\label{sec:fermi_nonrel}Non-relativistic ideal fermion gas}
This week we wrap up our applications of the grand-canonical ensemble to investigate ideal gases of non-interacting particles.
We again take the quantum statistical approach of defining micro-states by summing over the possible occupation numbers $n_{\ell}$ for each energy level $\cE_{\ell}$ with energy $E_{\ell}$.
In contrast to the bosonic case considered last week, we now focus on quantum gases of fermions, where the only possible occupation numbers are $n_{\ell} = 0$ and $1$, since the Pauli exclusion principle prevents multiple identical fermions from occupying the same energy level.

In \secref{sec:fermi} we derived the grand-canonical partition function (\eq{eq:partfunc_FD}) that defines quantum Fermi--Dirac statistics for such systems of non-interacting fermions,
\begin{equation*}
  \ZFD(\be, \mu) = \prod_{\ell = 0}^L \left[1 + e^{-\be (E_{\ell} - \mu)}\right],
\end{equation*}
for inverse temperature $\be = 1 / T$ and chemical potential $\mu$.
Recall that it is possible for systems of fermions to have any value for the chemical potential (either positive or negative), in contrast to the systems of bosons we considered last week.
From the corresponding grand-canonical potential,
\begin{equation*}
  \Phi_{\text{FD}} = -T \sum_{\ell = 0}^L \log\left[1 + e^{-\be (E_{\ell} - \mu)}\right]
\end{equation*}
we can determine the large-scale properties of the system, including its average internal energy $\vev{E}$, average particle number $\vev{N}$, entropy $S$, and pressure $P$, along with the equation of state relating these quantities.

A concrete calculation requires specifying the energy levels of the particles that compose the gas, and the degeneracies of those energy levels.
Let's begin this week by considering non-relativistic particles of the sort we previously analyzed in \secref{sec:regulate}.
In a volume $V = L^3$, the energy levels are defined by the non-zero quantized energies
\begin{align*}
  E(k) & = \frac{p^2}{2m} = \frac{\hbar^2 \pi^2}{2mL^2}\left(k_x^2 + k_y^2 + k_z^2\right) &
  k_{x, y, z} & = 1, 2, \cdots.
\end{align*}
In addition to the usual degeneracies coming from permutations of $(k_x, k_y, k_z)$ that we discussed in previous weeks, for each distinct $\vec k$ typical fermions such as electrons have two degenerate energy levels with the same energy.
This factor of $2$ has a different origin compared to the double degeneracy discussed for photons last week.
Rather than worry about the physical origins of this behaviour, in both cases we simply incorporate this given information into our ansatz. % TODO: Could mention spin...

The grand-canonical potential for an ideal gas of non-relativistic fermions is therefore
\begin{equation*}
  \Phi_{\text{f}} = T \sum_{\ell = 0}^L \log\left[1 + e^{-\be (E_{\ell} - \mu)}\right] = 2T \sum_{\vec k} \log\left[1 + \exp\left(-\frac{\hbar^2 \pi^2 k^2}{2mL^2 T} + \frac{\mu}{T}\right)\right].
\end{equation*}
We can again proceed by considering the gas in a large volume and approximating the sum over discrete integer $k_{x, y, z}$ by integrals over continuous real $\khat_{x, y, z}$:
\begin{equation*}
  \Phi_{\text{f}} \approx 2T \int d\khat_x d\khat_y d\khat_z \log\left[1 + \exp\left(-\frac{\hbar^2 \pi^2 \khat^2}{2mL^2 T} + \frac{\mu}{T}\right)\right].
\end{equation*}
Converting to spherical coordinates and carrying out the angular integrations over the $\frac{\pi}{2}$ solid angle of the octant of the sphere with $k_{x, y, z} > 0$, we have
\begin{equation*}
  \Phi_{\text{f}} \approx \pi T \int_0^{\infty} d\khat \; \khat^2 \log\left[1 + \exp\left(-\frac{\hbar^2 \pi^2 \khat^2}{2mL^2 T} + \frac{\mu}{T}\right)\right].
\end{equation*}
In the same spirit as the change of variables we carried out last week, to integrate over photon frequencies $\om = \Eph / \hbar$, we will now change variables to integrate over the fermion energy:
\begin{align*}
  E = \frac{\hbar^2 \pi^2}{2mL^2}\khat^2 \quad \lra \quad \khat & = \frac{L\sqrt{m}}{\pi \hbar} \sqrt{2E} \cr
                                                         d\khat & = \frac{L\sqrt{m}}{\pi \hbar} \frac{dE}{\sqrt{2E}}.
\end{align*}
Plugging this in produces
\begin{align}
  \Phi_{\text{f}} & \approx \pi T \left(\frac{L^3 m^{3 / 2}}{\pi^3 \hbar^3}\right) \int_0^{\infty} dE \frac{2E}{\sqrt{2E}} \log\left[1 + e^{-\be(E - \mu)}\right] \cr
                  & = \frac{\sqrt{2 m^3} VT}{\pi^2 \hbar^3} \int_0^{\infty} dE \sqrt{E} \log\left[1 + e^{-\be(E - \mu)}\right]. \label{eq:fermi_grand}
\end{align}
recognizing $L^3 = V$.

With this grand-canonical potential derived, we just need to take the appropriate derivatives to determine the thermodynamics and equation of state for non-relativistic fermions.
When doing so, we'll focus on the low-temperature regime where we expect quantum Fermi--Dirac statistics to differ significantly from the classical case we considered back in \secref{sec:regulate}.
As we saw in \secref{sec:quantum_classical}, at high temperatures (with large negative chemical potential) the classical approach provides a good approximation to the true quantum physics.
% TODO: Low temperatures provide a significant simplification, which is why we only consider that regime this week
% ------------------------------------------------------------------



% ------------------------------------------------------------------
\subsection{Low-temperature equation of state}
Rather than the average internal energy $\vev{E}_{\text{f}}$, it will prove profitable to first analyze the average particle number
\begin{equation*}
  \vev{N}_{\text{f}} = -\pderiv{}{\mu} \Phi_{\text{f}}
\end{equation*}
coming from the grand-canonical potential for non-relativistic fermions, \eq{eq:fermi_grand}.
In analogy to the Planck spectrum we derived for the photon gas last week, we first express the particle number density as an integral over energies,
\begin{equation*}
  \frac{\vev{N}_{\text{f}}}{V} = \frac{\sqrt{2m^3}}{\pi^2 \hbar^3} \int_0^{\infty} F(E) \sqrt{E} \; dE,
\end{equation*}
where the function $F(E)$ is known as the Fermi function.
In contrast to the Planck spectrum, all the constant factors are kept separate from $F(E)$:
\begin{align}
  \frac{\vev{N}_{\text{f}}}{V} & = \frac{\sqrt{2m^3} T}{\pi^2 \hbar^3} \int_0^{\infty} dE \sqrt{E} \pderiv{}{\mu} \log\left[1 + e^{-\be(E - \mu)}\right] \cr
                               & = \frac{\sqrt{2m^3}}{\pi^2 \hbar^3} \int_0^{\infty} \frac{1}{e^{\be(E - \mu)} + 1} \sqrt{E} \; dE = \frac{\sqrt{2m^3}}{\pi^2 \hbar^3} \int_0^{\infty} F(E) \sqrt{E} \; dE. \label{eq:N_fermi}
\end{align}
This allows the Fermi function to closely resemble the average occupation numbers $\vev{n_{\ell}}$ we computed in \secref{sec:quantum_classical}:
\begin{equation}
  \label{eq:Fermi_func}
  F(E) = \frac{1}{e^{\be(E - \mu)} + 1}.
\end{equation}

As usual in the grand-canonical approach, the average particle number and Fermi function depend on both the inverse temperature \be and the chemical potential $\mu$.
Expressing $F(E)$ in terms of the dimensionless ratios $E / \mu$ and $T / \mu$,
\begin{equation*}
  F(E) = \frac{1}{\exp\left[\frac{E - \mu}{T}\right] + 1} = \frac{1}{\exp\left[\frac{\mu}{T}\left(\frac{E}{\mu} - 1\right)\right] + 1} = \frac{1}{\left(\exp\left[\frac{E}{\mu} - 1\right]\right)^{\mu / T} + 1},
\end{equation*}
we can highlight the two main features of the figure below, which plots the Fermi function against $E / \mu$ for various temperatures $T / \mu$.

\begin{center}\includegraphics[width=\textwidth]{figs/week09_dist.pdf}\end{center}

First, we can see that the point $E = \mu$, where $F(E) = 1 / 2$ for any non-zero temperature, is a threshold at which the behaviour of the Fermi function changes.
For larger energies $E > \mu$, the exponential factor $\exp\left[\frac{E}{\mu} - 1\right] > 1$ and drives $F(E) \to 0$ as the energy increases.
For smaller energies $E < \mu$, the exponential factor $\exp\left[\frac{E}{\mu} - 1\right] < 1$ and vanishes as the energy decreases, leaving $F(E) \to 1$.
These two asymptotic limits reflect the possible energy level occupation numbers for fermions, $n_{\ell} = 0$ and $1$.
Second, smaller temperatures cause much more rapid approach to these two limits, with the exponential factor either enhanced (if $E > \mu$) or suppressed (if $E < \mu$) by a power $\mu / T \gg 1$.
Therefore, for small temperatures $T \ll \mu$, we can simplify our calculations by approximating the Fermi function as a step function,
\begin{equation}
  \label{eq:Fermi_step}
  F(E) \approx \left\{\begin{array}{l}1 \qquad \mbox{for } \ E < \mu \\
                                      0 \qquad \mbox{otherwise}\end{array}\right. .
\end{equation}
Using this approximation, what is the resulting particle number density?
\begin{mdframed}
  $\displaystyle \frac{\vev{N}_{\text{f}}}{V} = \frac{\sqrt{2m^3}}{\pi^2 \hbar^3} \int_0^{\infty} F(E) \sqrt{E} \; dE = $ \\[100 pt]
\end{mdframed}

You should find a result proportional to $\mu^{3 / 2}$ but independent of $T$.
The temperature independence can be understood by viewing this as the leading-order term in an expansion in the small temperature (known as a \textit{Sommerfeld expansion}, named after \href{https://en.wikipedia.org/wiki/Arnold_Sommerfeld}{Arnold Sommerfeld}).
The $\mu^{3 / 2}$ dependence on the chemical potential is also what we would predict even without doing the detailed calculation.
The step function in \eq{eq:Fermi_step} corresponds to a single fermion occupying each and every energy level with $E_{\ell} < \mu$, while all energy levels with $E_{\ell} > \mu$ are unoccupied.
Since $E(k) \propto k^2$, summing over all $k_{x, y, z}$ corresponds to computing (a portion of) the volume of a sphere of radius $r = \sqrt{\mu}$, which is proportional to $r^3 = \mu^{3 / 2}$ as we found directly above.
If we were to invert this relation, we would obtain the so-called \textbf{Fermi energy} as a function of the particle number density,
\begin{equation}
  \label{eq:Fermi_energy}
  E_F = \mu = \frac{\hbar^2}{2m}\left(\frac{3\pi^2 \vev{N}_{\text{f}}}{V}\right)^{2 / 3}.
\end{equation}

Now we can consider the average energy density of the non-relativistic fermion gas at low temperatures.
Rather than taking another derivative of the grand-canonical potential, we can note from \eq{eq:total_energy_levels} and from our work on the photon gas last week that
\begin{equation}
  \frac{\vev{E}_{\text{f}}}{V} = \frac{\sqrt{2m^3}}{\pi^2 \hbar^3} \int_0^{\infty} E \; F(E) \sqrt{E} \; dE.
\end{equation}
That is, instead of simply counting the number of fermions in the system, we need to add up their energies, introducing an extra factor of $E$ compared to \eq{eq:N_fermi}.
Still using the low-temperature step-function approximation for the Fermi function in \eq{eq:Fermi_step}, what is the average energy density?
\begin{mdframed}
  $\displaystyle \frac{\vev{E}_{\text{f}}}{V} = \frac{\sqrt{2m^3}}{\pi^2 \hbar^3} \int_0^{\infty} F(E) E^{3 / 2} \; dE = $ \\[100 pt]
\end{mdframed}

You should find
\begin{equation}
  \label{eq:fermi_E_N}
  \vev{E}_{\text{f}} = \frac{3}{5} \mu \vev{N}_{\text{f}},
\end{equation}
which means that the average energy of the fermions in a low-temperature ideal gas is $3 / 5$ the Fermi energy $E_F = \mu$.
In particular, because this result is also independent of the temperature, we find that non-interacting quantum fermions retain a positive energy even as the temperature approaches absolute zero, $T \to 0$.
This can be understood by recalling that the lowest-energy pair of degenerate energy levels can each only hold a single fermion, forcing all additional fermions to `fill' energy levels with larger energies $E_{\ell} > 0$, up to the Fermi energy set by the chemical potential.
This is a stark contrast to the classical ($\vev{E} \propto T$) and bosonic ($\vev{E} \propto T^4$) cases we considered earlier, where the average energy vanishes in the zero-temperature limit.
As discussed in Sections~\ref{sec:spin_info} and \ref{sec:quantum_classical}, in those cases all the particles in the system are able to fall into the lowest energy level, with only an exponentially small probability for particles to occupy any energy levels with $E_{\ell} > E_0$.

To get the rest of the way to the equation of state for the ideal gas of non-relativistic fermions, we need to compute the pressure
\begin{equation*}
  P_{\text{f}} = -\left. \pderiv{}{V} \vev{E}_{\text{f}} \right|_{N, S_{\text{f}}}.
\end{equation*}
In the low-temperature limit, the condition of constant entropy $S_{\text{f}} = -\sum_{i = 1}^M p_i \log p_i$ is automatically satisfied, since the step function in \eq{eq:Fermi_step} restricts the system to a single micro-state, resulting in $S_{\text{f}} = 0$.
This micro-state is the one in which each and every energy level with $E_{\ell} < \mu$ is occupied, while all energy levels with $E_{\ell} > \mu$ are unoccupied.

Inserting \eq{eq:Fermi_energy} into \eq{eq:fermi_E_N}, we have
\begin{equation*}
  \vev{E}_{\text{f}} = \frac{3}{5} \mu \vev{N}_{\text{f}} = \frac{3}{5} \left(\frac{\hbar^2}{2m}\right) \left(\frac{3\pi^2}{V}\right)^{2 / 3} \vev{N}_{\text{f}}^{5 / 3}.
\end{equation*}
The pressure is therefore
\begin{align}
  P_{\text{f}} & = -\pderiv{}{V} \left[\frac{3}{5} \left(\frac{\hbar^2}{2m}\right) \left(\frac{3\pi^2}{V}\right)^{2 / 3} \vev{N}_{\text{f}}^{5 / 3}\right] = \frac{2}{3V} \left[\frac{3}{5} \left(\frac{\hbar^2}{2m}\right) \left(\frac{3\pi^2}{V}\right)^{2 / 3} \vev{N}_{\text{f}}^{5 / 3}\right] \nonumber \\
               & = \left(3\pi^2\right)^{2 / 3} \frac{\hbar^2}{5m} \left(\frac{\vev{N}_{\text{f}}}{V}\right)^{5 / 3} = \frac{2}{5} \mu \frac{\vev{N}_{\text{f}}}{V} = \frac{2}{3} \frac{\vev{E}_{\text{f}}}{V}.
\end{align}
The three expressions in the second line above present several relations between the pressure, the energy density, the Fermi energy $E_F = \mu$ and the particle number density.
In particular, we can see that the pressure (like the energy) remains positive even as the temperature approaches absolute zero, with
\begin{equation}
  \label{eq:degen_pressure}
  P_{\text{f}} = \left(3\pi^2\right)^{2 / 3} \frac{\hbar^2}{5m} \rho_{\text{f}}^{5 / 3},
\end{equation}
where we define the density $\rho_{\text{f}} = \vev{N}_{\text{f}} / V$.
This positive pressure in the zero-temperature limit is not due to any direct force between the fermions, which remain non-interacting in this ideal gas.
Instead, it is a purely quantum effect resulting from the Pauli exclusion principle.

As we saw earlier in this section, the temperature independence of the pressure $P_{\text{f}}$ is due to approximating the low-temperature Fermi function as a step function in \eq{eq:Fermi_step}, and systematic corrections to this approximation can be computed through a Sommerfeld expansion.
Even without getting into such detailed calculations, we know that at high temperatures the quantum ideal gas of massive fermions will be well approximated by the classical ideal gas we considered in \secref{sec:ideal_gas}, with equation of state
\begin{equation}
  PV = NT \qquad \Lra \qquad P = \frac{N}{V} T = \rho T.
\end{equation}
In words, at high temperatures the pressure depends linearly on the temperature, with the slope corresponding to the density $\rho$.
The plot below (produced by \href{https://github.com/daschaich/MATH327_2021/blob/master/lecture_notes/figs/week09_pressure.py}{this Python code}) shows how the pressure changes from a positive constant as $T \to 0$ to this linear behaviour at higher temperatures. \\[-24 pt] % WARNING: FORMATTING BY HAND

\begin{center}\includegraphics[width=0.725\textwidth]{figs/week09_pressure.pdf}\end{center}
% ------------------------------------------------------------------



% ------------------------------------------------------------------
\subsection{Type-Ia supernovas}
The positive pressure that remains for a fermion gas even at zero temperature, \eq{eq:degen_pressure}, is known as the \textit{degeneracy pressure}.
(This use of the word `degenerate' is completely unrelated to its other use describing multiple energy levels with the same value of the energy.)
The degeneracy pressure plays a key role in a famous cosmic phenomenon---a certain class of supernova explosions of stars.

As a opening remark to this topic, note that the temperature doesn't need to be exactly zero in order for the degeneracy pressure to be significant.
The temperature just needs to be small compared to the Fermi energy, $T \ll E_F$.
From \eq{eq:Fermi_energy} we can see that $E_F \propto \rho_{\text{f}}^{2 / 3}$ increases for larger densities $\rho_{\text{f}} = \vev{N}_{\text{f}} / V$.
Not surprisingly, the densities of stars can be very large indeed, due to the enormous amount of matter that is being squeezed together by gravitational attraction.
Everyday matter has densities around $10^{28}$--$10^{29}$ atoms per cubic metre (roughly corresponding to Avogadro's number per cubic centimetre), which results in Fermi energies $E_F \sim 10^4$~K. % 2--10 eV with eV~10^4 K being the Boltzmann constant
Fermi energies for particularly dense stars known as \textit{white dwarfs} are a hundred thousand times larger, $E_F \sim 10^9$~K, corresponding to densities of roughly one tonne per cubic centimetre.
This is around a million times more dense than our sun---while white dwarf stars have a mass similar to our sun's $M_{\odot}$, their radius is a hundred times smaller, comparable to the radius of the earth.

White dwarf stars are so dense because they have exhausted the hydrogen and helium `fuel' for the nuclear fusion that generates heat and light---and therefore radiation pressure---in stars such as our sun.
For actively `burning' stars, this radiation pressure counteracts the gravitational attraction of the star's enormous mass.
Without nuclear fusion, white dwarfs end up gravitationally compressed into much denser and more compact objects.
The degeneracy pressure, \eq{eq:degen_pressure}, coming from the (fermionic) electrons in the white dwarf is what stabilizes these stars and prevents them from collapsing further into even denser objects such as black holes.

It is remarkable that even under these extreme conditions the electrons in white dwarf stars are well described by the non-interacting ideal fermion gas we have analyzed above.
In particular, it is crucial that white dwarfs' Fermi energies are so large, $E_F \sim 10^9$~K.
Even though white dwarfs have burned all their nuclear fuel, their interiors remain quite hot by everyday standards, roughly ten million kelvin ($T \sim 10^7$~K). % 0.3 MeV ~ 10^5 eV with eV~10^4 K being the Boltzmann constant
It is only due to their large densities and Fermi energies that $T \ll E_F$ and white dwarfs can be treated as zero-temperature objects to a good approximation.

So far we've seen no sign of supernovas.
In isolation, white dwarfs will happily cool for trillions of years, supported by their degeneracy pressure, until they reach thermal equilibrium with the $T_{\text{CMB}} \approx 2.725$~K cosmic microwave background radiation we considered last week. % TODO: CMB temperature will also keep decreasing as universe continues expanding...
Things become more interesting for a white dwarf that forms a binary system with a companion star.
If this companion star that is still burning hydrogen or helium through nuclear fusion, it will emit matter that is then captured by the white dwarf, slowly increasing the white dwarf's mass.
Such a binary system is pictured below, in an artist's illustration provided by the \href{https://www.esa.int/ESA_Multimedia/Images/2014/09/Artist_s_impression_of_Type_Ia_supernova}{European Space Agency}.

\begin{center}\includegraphics[width=0.8\textwidth]{figs/week09_nova.pdf}\end{center}

As the white dwarf accumulates the matter emitted by its companion, its mass and its density will steadily increase.
As the mass of the white dwarf approaches a value about 40\% larger than the mass of our sun (known as the Chandrasekhar limit, named after \href{https://en.wikipedia.org/wiki/Subrahmanyan_Chandrasekhar}{Subrahmanyan Chandrasekhar}), its density becomes large enough for new types of nuclear fusion reactions to occur.
Instead of hydrogen or helium, which the white dwarf has already burned, these new fusion reactions involve carbon and oxygen, which the white dwarf has in plenty.
In the space of just a few seconds, these fusion reactions run away, increase the temperature of the white dwarf to billions of kelvin, and blast it apart in a supernova explosion about five billion times brighter than the sun.

For obscure historical reasons, these particular stellar explosions are known as \textit{type-Ia} (``one-A'') \textit{supernovas}.
They rely on the degeneracy pressure (\eq{eq:degen_pressure}) of a low-temperature gas of non-interacting fermions, which allows a specific amount of mass to build up before the explosion is triggered.
The specificity of the process results in a great deal of regularity among type-Ia supernovas, which is very useful to astronomers.
In particular, these type Ia supernovas play a key role in demonstrating that the expansion of the universe is accelerating (a phenomenon popularly called `dark energy'), which was awarded the 2011 Nobel Prize in Physics.
% ------------------------------------------------------------------



% ------------------------------------------------------------------
\subsection{Relativistic ideal fermion gas}
While our main focus this week is on non-relativistic gases with $E \propto p^2$, gases of relativistic fermions also appear in nature.
In fact, by changing units we can see that the white dwarf Fermi energy discussed above, $E_F \sim 10^9~\text{K} \sim 0.3$~MeV is comparable to the $0.511$~MeV rest-energy of electrons, suggesting that relativistic effects may be non-negligible in white dwarfs.
Such relativistic effects were indeed taken into account by Chandrasekhar and others investigating these compact stars in the twentieth century.
While the full calculations required to analyze massive relativistic particles are beyond the scope of this module, we can benefit from last week's consideration of gases of massless photons to briefly consider similar gases of massless fermions.
\textit{Neutrinos} (denoted `$\nu$') are physical examples of particles whose masses are so small that they can be very well approximated as massless fermions.\footnote{Neutrinos' masses are so small that it was extremely difficult to determine that they are not exactly massless.  The 2015 Nobel Prize in Physics was awarded for this discovery.}

In the same way as photons, such massless fermions travel at the speed of light, $c$, and have energies $E = cp$ that depend on their angular frequencies,
\begin{equation}
  E_{\nu} = \hbar \om.
\end{equation}
In a volume $V = L^3$, these energies are quantized as usual,
\begin{equation*}
  \om = \frac{2\pi c}{\la} = c \frac{\pi}{L} k,
\end{equation*}
where $k^2 = k_x^2 + k_y^2 + k_z^2$ and $k_{x, y, z} > 0$ are positive integers.
Just as for the massive fermions considered in \secref{sec:fermi_nonrel}, for each distinct set of integers $\vec k = (k_x, k_y, k_z)$, typical massless fermions such as neutrinos have two degenerate energy levels with the same energy.

The detailed analysis of a gas of massless fermions is nearly the same as the work we did last week for photon gases.
In particular, massless fermions are also well described by a vanishing chemical potential, $\mu \approx 0$.
Again approximating sums over discrete integer $k_{x, y, z}$ by integrals over continuous real $\khat_{x, y, z}$, and changing variables to integrate over the angular frequency, we end up with the grand-canonical potential
\begin{align}
  \Phi_{\nu} = -\frac{VT}{c^3 \pi^2} \int_0^{\infty} d\om \; \om^2 \log\left[1 + e^{-\be \hbar \om}\right]. \label{eq:neutrino_grand}
\end{align}
The only change in $\Phi_{\nu}$ compared to \eq{eq:photon_grand} for photons are a couple of negative signs, precisely as we would expect from comparing the Bose--Einstein and Fermi--Dirac grand-canonical potentials in \secref{sec:quantum_classical}.

Carrying out the derivatives to obtain the average particle number density and the average internal energy density produces
\begin{align}
  \frac{\vev{E}_{\nu}}{V} & = \frac{1}{c^3 \pi^2} \int_0^{\infty} \frac{\hbar \om^3}{e^{\be \hbar \om} + 1} d\om &
  \frac{\vev{N}_{\nu}}{V} & = \frac{1}{c^3 \pi^2} \int_0^{\infty} \frac{\om^2}{e^{\be \hbar \om} + 1} d\om,
\end{align}
now differing only by negative signs in their denominators compared to the photon gas results we obtained last week.
The condition of constant entropy is still $V T^3 = \mbox{constant}$, and the resulting pressure leads to an equation of state in the usual form,
\begin{equation*}
  P_{\nu} V \propto \vev{N}_{\nu} T,
\end{equation*}
with the constant of proportionality another $\cO(1)$ number involving the Riemann zeta function.
% ------------------------------------------------------------------
