% ------------------------------------------------------------------
\newcommand{\thisunit}{MATH327 information}
\newcommand{\moddate}{Last modified 26 Jan.~2022}
\setcounter{section}{0}
\phantomsection
\addcontentsline{toc}{section}{Module information and logistics}
\section*{Module information and logistics}

\subsection*{Coordinator}
\begin{description}
  \setlength{\itemsep}{1pt}
  \setlength{\parskip}{0pt}
  \setlength{\parsep}{0pt}
  \item[\qquad] David Schaich
  \item[\qquad] Mathematics Building Room 123 (Theoretical Physics Wing)
  \item[\qquad] \href{mailto:david.schaich@liverpool.ac.uk}{david.schaich@liverpool.ac.uk} \hfill \href{http://www.davidschaich.net}{www.davidschaich.net}
\end{description}
% ------------------------------------------------------------------



% ------------------------------------------------------------------
\subsection*{Overview}
Along with quantum mechanics and relativity, statistical physics is a pillar of modern physics.
It uses stochastic (i.e., probabilistic) techniques to predict the large-scale behaviour that emerges from the microscopic dynamics of \emph{many} independent objects.
(Think of the $\sim$$10^{22}$ atoms in a gram of water.)
Statistical physics find applications from nuclear physics and cosmology to climate and biological physics, often with outstanding success.

The module outline on the previous page is organized around the concept of \textit{statistical ensembles} introduced in the early 1900s.
In essence, a statistical ensemble is a mathematical framework for concisely describing the properties of idealized physical systems subject to certain constraints.
After studying the probability foundations underlying these frameworks, we meet the \textit{micro-canonical ensemble} in unit 2 and the \textit{canonical ensemble} in unit 3.
The following two units 4--5 apply the canonical ensemble to investigate non-interacting (``ideal'') gases and thermodynamic cycles.
Unit 6 introduces our third and final statistical ensemble, the \textit{grand-canonical ensemble}, which units 7--8 apply to various types of non-interacting quantum gases.
Finally, in unit 9 we begin to explore the effects of interactions, which open up a much broader landscape of applications that we will survey for the remainder of the term.
% ------------------------------------------------------------------



% ------------------------------------------------------------------
\subsection*{Schedule and planned operating procedure}
Pandemic permitting, we will meet in person at 16:00 on Mondays, 9:00 on Tuesdays \& Fridays, and 13:00 on Fridays.
While these are labeled as lectures \& tutorial on the timetable, the lectures in particular won't be identical to the pre-pandemic state of affairs: All core materials will continue to be available asynchronously in advance, and our in-person meetings will place greater emphasis on active learning and discussion.
Each of these meetings will also connect to Zoom at \href{https://liverpool-ac-uk.zoom.us/j/92984060744?pwd=WkVPek8ySlVGVmZVQXJSL21HTTd2Zz09}{this link} (meeting ID 929 8406 0744, passcode MATH327!) for those who are not able to come on to campus (e.g., due to isolating).
Zoom recordings will be collected along with all other resources at \href{https://liverpool.instructure.com/courses/47333}{our Canvas site}, \\
\centerline{\href{https://liverpool.instructure.com/courses/47333}{liverpool.instructure.com/courses/47333}}
% I will assume announcements and other Canvas communications will be seen within 24 hours.

\textbf{Office hours} will take place at 17:00 on Mondays and 10:00 on Fridays, immediately after the corresponding lectures.
You can either visit Room 123 in the Theoretical Physics Wing, or connect through the same Zoom meeting linked above.
If these times do not work with your schedule, you can also make an appointment through \href{https://calendly.com/daschaich/meet}{calendly.com/daschaich}, or use the \href{https://liverpool.instructure.com/courses/47333/discussion_topics}{Canvas discussion board}.

%These plans may need adjustment going forward, so we'll remain flexible as the term progresses.
% ------------------------------------------------------------------



% ------------------------------------------------------------------
\subsection*{Assessment and academic integrity}
The following deadlines for in-term assignments have been coordinated within the Department to minimize pile-up.
However, the assessment coordinator \textbf{warns}: ``Some students (primarily those taking unusual module combinations) will need to manage their time quite carefully at particular points.''
\begin{description}
  \item[20\%] A computer project divided into two equally weighted parts, the first due \textbf{Thursday, 17 February} and the second due \textbf{Thursday, 24 March}
  \item[30\%] Two equally weighted homework assignments, the first due \textbf{Thurs., 3 March} and the second due \textbf{Thursday, 5 May}
  \item[50\%] A one-hour in-person final examination to be centrally scheduled within the period 16 May through 3 June
\end{description}

According to the University's \href{https://www.liverpool.ac.uk/media/livacuk/tqsd/code-of-practice-on-assessment/code_of_practice_on_assessment.pdf}{Code of Practice on Assessment} (COPA), late submissions completed within five days (120 hours) after the submission deadline will have 5\% of the total marks deducted for each 24-hour post-deadline period.
Submissions more than five days late will be awarded zero marks, though I will still endeavour to provide feedback on them.
I will aim to return feedback and solutions 7--10 days after the deadline.
In particular, you should receive feedback from the first part of the computer project well before the deadline for the second part, and all feedback should be returned well before the final exam.

By now you should have successfully passed the Academic Integrity Tutorial and Quiz to affirm that you have read and understood the \href{https://www.liverpool.ac.uk/media/livacuk/tqsd/code-of-practice-on-assessment/appendix_L_cop_assess.pdf}{Academic Integrity Policy} detailed in COPA Appendix L.
If you have any questions about what is or is not acceptable under this policy, please ask me or Departmental Assessment Officer Kamila Zychaluk.
In all cases, the spirit of the Academic Integrity Policy should take precedence over legalistic convolutions of the text.

In particular, I encourage you to discuss the in-term assignments with each other, since discussing and debating concepts and procedures is a very effective way to learn.
The examination must be done on your own, and your submissions for all assignments must be your own work representing your own understanding.
Clear and neat presentations of your workings and the logic behind them will contribute to your mark.
It is unacceptable to copy solutions in part or in whole from other students (current or prior) or from other sources (commercial or otherwise).
Should you make use of resources beyond the module materials, these must be explicitly referenced in your work.
% ------------------------------------------------------------------



% ------------------------------------------------------------------
\subsection*{Main resources and materials}
The main materials we will use are the lecture notes you are currently reading.
As you read further, you will encounter \textbf{gaps} in the notes where you will be able to check your understanding by completing some exercises that are intended to be bite-sized.
We will go over these gaps at our in-person meetings, but I encourage you to think about them for yourself.

While these lecture notes have been developed over the previous two years, they continue to be improved, expanded and sometimes corrected.
The ``Last modified'' date at the bottom of each page will flag any changes that occur during the term.
You can track the changes themselves through the public version control repository for these notes: \\
\centerline{\href{https://github.com/daschaich/MATH327_2022}{github.com/daschaich/MATH327\_2022}}
This repository also allows you to optionally contribute to the co-creation of the module, by raising issues with the notes and creating pull requests to address them.
The \href{https://software-carpentry.org}{Software Carpentry} project provides an introduction to \href{https://swcarpentry.github.io/git-novice/}{Version Control with Git}.

We work in units where the Boltzmann constant $k = 1$, and logarithms have base $e$ unless otherwise specified (i.e., $\log x = \ln x$).
There is no need to memorize any equations.
Many equations are numbered so that they can be referenced later on, not necessarily because they are important.
Key results, definitions and concepts are highlighted by coloured boxes, and you should be confident in your understanding of these.

\subsubsection*{Expected background}
Familiarity with quantum mechanics or computer programming is \textbf{not} assumed.
Any necessary information on these topics will be provided.
I do anticipate that you have previously seen the \href{https://en.wikipedia.org/wiki/Standard_deviation}{standard deviation}, the \href{https://en.wikipedia.org/wiki/Binomial_coefficient}{binomial coefficient}
\begin{equation*}
  \binom{N}{k} = \frac{N!}{k! \, (N - k)!} = \binom{N}{N - k}
\end{equation*}
that counts the number of possible ways to choose $k$ objects out of a set of $N \geq k$ total objects, and \href{https://en.wikipedia.org/wiki/Gaussian_integral}{gaussian integrals},
\begin{equation*}
  \int_{-\infty}^{\infty} e^{-a (x + b)^2} \; dx = \sqrt{\frac{\pi}{a}}.
\end{equation*}

\subsubsection*{Programming}
You are welcome to complete the computer project using the programming language of your choice.
I recommend \href{https://www.python.org}{Python}, which is free, user-friendly, and very widely used around the world.
In tutorials we will discuss \href{https://tinyurl.com/math327demo}{this demo} that illustrates all the Python programming tools you'll need.
Python may already be available on devices you have access to---either a personal computer or computers in our library.
If not, you can also run Python code on many Web sites including \href{https://replit.com/languages/python3}{replit.com}, \href{https://trinket.io/python3}{trinket.io},
and \href{https://colab.research.google.com}{Google Colaboratory}. % onlinegdb.com doesn't seem to handle figure display or saving; mybinder.org requires git repo
You may need to create a free account, and you should make sure to save a local copy so your work isn't lost.
%
Alternate languages could include \href{https://en.wikipedia.org/wiki/C_(programming_language)}{C}, \href{https://www.r-project.org}{R}, or even \href{https://matlab.mathworks.com}{MATLAB} (through the University's site license).
Maple may struggle to handle parts of the project.
% ------------------------------------------------------------------



% ------------------------------------------------------------------
\subsection*{How to get the most out of this module}
At this point in your studies, this advice should all be familiar, but there's no harm in repeating it.

Come to class, or sign in synchronously if isolating.
This will provide regular contact with the material, and help make sure you understand it.
If the module is moving slower than you'd like, coming to class will give you opportunities to ask about more interesting extensions, applications or complications.

Look through the lecture notes in advance, and try to fill in the gaps.
Aim to identify the overarching logic rather than digging in to every detail, and note any questions (or apparent over-simplifications) that should be addressed in class.
Even though the lecture notes reflect my plans for the module, they may not correspond to what happens in class in the end.
We may gloss over some topics that seem to be sufficiently well explained in the notes, and we may delve deeper into other topics that seem to merit further consideration.

Work on the homework problems, computer project and tutorial exercises.
The best way to learn mathematics is by doing mathematics, and these assignments are designed to make you think and further develop your mathematical muscles.
In particular, the homework problems will be harder than exam questions, since you'll have much more time to work on them---so make sure you start thinking about them well in advance of the deadline.
Afterwards, review the solutions and any feedback, to make sure any confusing points are resolved.

Ask questions.
Ask questions you think you're supposed to know the answer to.
Ask questions you think everyone else knows the answer to (they don't).
Ask questions about the big picture, or the individual details, or the connections between them.
The opportunity to ask questions is the main benefit of taking a module.
You can ask me; you can ask your classmates; you can ask the additional resources below.
% If you ask me about assignments, I will avoid doing the work for you; instead I will have you explain to me what you have tried so far, and will ask leading questions to suggest where I see problems or potential next steps.

\subsubsection*{Additional resources}
The lists below provide some optional additional resources that may be helpful.
You can use the module \href{https://liverpool.instructure.com/courses/47333/external_tools/102}{Reading List} on Canvas to reach our library's records for the books.

\noindent\textbf{Books and lecture notes at roughly the level of this module:} \\[-24 pt]
\begin{enumerate}
  \item David Tong, \href{https://www.damtp.cam.ac.uk/user/tong/statphys.html}{\textit{Lectures on Statistical Physics}} (2012), \\ www.damtp.cam.ac.uk/user/tong/statphys.html
  \item Daniel V.~Schroeder, \textit{An Introduction to Thermal Physics} (2021)
  \item C.~Kittel and H.~Kroemer, \textit{Thermal Physics} (1980)
  \item F.~Reif, \textit{Fundamentals of Statistical and Thermal Physics} (1965)
\end{enumerate}

\newpage % WARNING: FORMATTING BY HAND
\noindent\textbf{More advanced and more specialized books}, which may be useful to consult concerning specific questions or topics: \\[-24 pt]
\begin{enumerate}
  \setcounter{enumi}{4}
  \item R.~K.~Pathria, \textit{Statistical Mechanics} (1996)
  \item Sidney Redner, \textit{A Guide to First-Passage Processes} (2007)
  \item Pavel L.~Krapivsky, Sidney Redner and Eli Ben-Naim, \textit{A Kinetic View of Statistical Physics} (2010)
  \item Kerson Huang, \textit{Statistical Mechanics} (1987)
  \item Weinan E, Tiejun Li and Eric Vanden-Eijnden, \textit{Applied Stochastic Analysis} (2019)
  \item Michael Plischke \& Birger Bergersen, \textit{Equilibrium Statistical Physics} (2005)
  \item L.~D.~Landau and E.~M.~Lifshitz, \textit{Statistical Physics, Part 1} (1980)
\end{enumerate}

\noindent\textbf{A general book about learning}, which (among other strategies) emphasizes the value of retrieval practice compared to re-reading lecture notes or re-watching videos: \\[-24 pt]
\begin{enumerate}
  \setcounter{enumi}{11}
  \item Peter C.~Brown, Henry L.~Roediger III and Mark A.~McDaniel, \textit{Make it Stick: The Science of Successful Learning} (2014) \\
        A \href{https://www.youtube.com/watch?v=MfylloWuuZU}{short summary video} is also available
\end{enumerate}

\noindent\textbf{Programming resources:} \\[-24 pt]
\begin{enumerate}
  \setcounter{enumi}{12}
  \item MATH327 \href{https://tinyurl.com/math327demo}{Python programming demo} (2022)
  \item \href{https://wiki.python.org/moin/BeginnersGuide}{Beginner's Guide to Python} (2022)
  \item \href{https://software-carpentry.org}{Software Carpentry} tutorials: \\
        \textcolor{white}{hack} \href{https://swcarpentry.github.io/git-novice/}{Version Control with Git} (2022) \\
        \textcolor{white}{hack} \href{https://swcarpentry.github.io/python-novice-inflammation/}{Programming with Python} (2022) \\
        \textcolor{white}{hack} \href{http://swcarpentry.github.io/python-novice-gapminder/}{Plotting and Programming in Python} (2022)
  \item Stormy Attaway, \textit{MATLAB: A Practical Introduction to Programming and Problem Solving} (2013)
  \item B.~Barnes and G.~R.~Fulford, \textit{Mathematical Modelling with Case Studies: Using Maple and MATLAB} (2014)
\end{enumerate}

In addition, there is a vast constellation of other online resources such as \href{https://en.wikipedia.org/wiki/Statistical_physics}{Wikipedia}.
These are often fine places to \emph{start} learning about a topic, but may be terrible places to \emph{stop}.
% ------------------------------------------------------------------
